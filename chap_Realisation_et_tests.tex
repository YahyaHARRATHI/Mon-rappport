\chapter*{Réalisation et tests}

\setcounter{page}{0}
\thispagestyle{empty}

\newpage

%================header and footer============

\fancyhf{}% Clear header/footer
\newpagestyle{ruled}
{\sethead{}{}{Réalisation et tests}\headrule
  \setfoot{ISI : Institut Supéerieur d'Informatique}{}
  { \thepage}\footrule}
\pagestyle{ruled}

\renewcommand\makeheadrule{\color{cyan}\rule[-.3\baselineskip]{\linewidth}{2.5pt}}
\renewcommand\makefootrule{\color{cyan}\rule[\baselineskip]{\linewidth}{2.5pt}}


%==============Début
\begin{Large}
Introduction 
\end{Large}

Après avoir élaboré la conception de mon application, j’aborde dans ce chapitre le dernier volet de ce rapport, qui a pour objectif d'exposer la phase de réalisation.\\
Je mène tout d’abord une étude technique où je décris les ressources logicielles utilisées dans le développement du projet.\\
 Je présente en premier lieu mon choix de l’environnement de travail, où je spécifie l’environnement matériel et logiciel que j’ai utilisé pour réaliser l’application puis je détaille l’architecture.\\
Par la suite, je présente quelques interfaces réalisées pour illustrer le fonctionnement de quelques activités du système. 

%\renewcommand{\thesection}{\arabic{section}}
\section{Etude technique}

\subsection{Environnement matériel}
A fin de réaliser le projet, j’ai travaillé sur un ordinateur portable qui a les caractéristiques suivantes :\\ 

\begin{table}[!h]
\begin{center}
\begin{tabular}{|l|l|}
\hline 
Marque & DELL INSPIRON N5040 \\ 
\hline 
Processeur & Intel core i3 @2.20GHZ \\ 
\hline 
Mémoire & 6,00 Go \\ 
\hline 
Système d'exploitation & Windows 8.1 64 bits \\ 
\hline 
\end{tabular} 
\end{center}
\caption{Tableau de l'environnement matériel}
\end{table}

\subsection{Environnement logiciel}
\subsubsection{Outils de développement}





















